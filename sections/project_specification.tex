\chapter{Project Specification}

 The aim of this project is to design and build a product that resolves the problems with the original Cuisenaire\textsuperscript{\textregistered} Rods outlined in Chapter \ref{sec:intro}, by using a technologically-enhanced version of the rods which will allow the teacher to detect struggling pupils and provide assistance to them, as well as keeping an electronic record of all the students' sessions with the rods. They should do so by gathering data about the way the students are using the enhanced rods, such as time spent on an exercise, how many answers were found, and perhaps even more complex information like detecting patterns in which solutions were discovered. This data can be analysed to provide an overview of how the entire class is coping with the task.\\

The project is split into two halves: the software, including processing of data and user interfaces, is being completed by another student (Pierre Azalbert). The hardware, including the design and production of the rods, is the focus of this report.\\


Making the correct design choices will greatly affect the efficacy of the product. The final product will likely be used by publicly-funded schools, so costs should be kept low. The aim of the product is to make the teaching process easier for the teacher and to improve the learning experience for the pupils, so care should be taken to make the product as accessible as possible for both parties. This means keeping setup steps simple for teachers, and using human-computer interface (HCI) principles to ensure students find the product easy to use. 

%%%%To reiterate, every design decision made should be in the aid of either making the 

