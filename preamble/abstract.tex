\section*{Abstract}

This project explores the use of technology to enhance the efficacy of educational tools, specifically, the use of Cuisenaire\textsuperscript{\textregistered} Rods in Key Stage One mathematics classrooms. These rods are coloured cuboids of different lengths that are stacked beside each other by the student in order to create a physical representation of the addition of numbers. The project is split into two halves: the software, including processing of data and user interfaces, is being completed by another student (Pierre Azalbert). The hardware, including the design and production of the rods, is the focus of this report. The report outlines the design choices made, with respect to manufacturing methods, materials used, and chosen technologies, comparing the suitability of different viable methods. The completed product consists of a 3D-printed playing board containing chains of resistors, upon which rods are placed by the user. The rods contain a conductive wire that, when connected to the board via magnetic contacts, will short a certain length of the chain, depending on the rod's length. An Arduino microcontroller in the board detects the length of rod that has been placed on the board, and transmits this data to a web server where the information is displayed for the convenience of the teacher, allowing them to monitor and track student progress during classroom activities. The enhanced version of the rods improves upon the original as it reduces the need for manual recording of children’s work, which takes valuable time away from the learning exercise.