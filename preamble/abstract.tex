\section*{Abstract}

This project explores the use of technology to enhance the effectiveness of educational tools, specifically, the use of Cuisenaire\textsuperscript{\textregistered} Rods in Key Stage One mathematics classrooms. These rods are coloured cuboids of different lengths that are stacked beside each other by the child to create a physical representation of  addition. Another student,  P. Azalbert, is responsible for data processing and user interfaces, whereas the focus of this report is the hardware, including the design and production of enhanced rods and the choices made with respect to manufacturing methods and technologies used. The completed product consists of a 3D-printed playing board containing chains of resistors, upon which rods are placed by the user. The rods contain a conductive wire that, when connected to the board via magnetic contacts, will short a certain length of the chain, depending on the rod's length. Information about what rods have been placed on the board is transmitted to a server and recorded for the teacher, allowing them to track student progress during classroom activities. The enhanced version of the rods improves upon the original as it reduces the need for manual recording of children’s work, which takes valuable time away from the learning exercise.